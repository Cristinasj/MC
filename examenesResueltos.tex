\documentclass[12pt]{article}
\usepackage[utf8]{inputenc}

\title{Preparación exámen MC. Ejercicios resueltos de examen.}
\author{Cristina Sanchez}
\date{15 enero 2022}
\usepackage{tikz}
\def\checkmark{\tikz\fill[scale=0.4](0,.35) -- (.25,0) -- (1,.7) -- (.25,.15) -- cycle;} 

\begin{document}

\maketitle

\newpage

\section{Introduction}
En este documento podéis encontrar resumenes de teoría y ejercicios de examen resueltos de cosas que pueden entrar en el exámen de MC. Espero que os sirva de ayuda. 
\newline
Los ejercicios que están añadidos de momento son los que tienen el símbolo $\checkmark$ 
Esta es la versión 0. Seguiré ampliando el documento hasta el día del exámen. Mirad siempre la última versión. 

\subsection{Contenidos:} 
\subsubsection{Conversiones}
\begin{itemize}
    \item Lenguaje $\rightarrow$ Autómata
    \item Lenguaje $\rightarrow$ ER $\checkmark$
    \item Lenguaje $\rightarrow$ Gramática 
    \item ER $\rightarrow$ Lenguaje 
    \item ER $\rightarrow$ Autómata 
    \item ER  $\rightarrow$  Gramática  
    \item Autómata  $\rightarrow$  lenguaje 
    \item Autómata  $\rightarrow$  ER
    \item Autómata  $\rightarrow$ Gramática
    \item Gramáticas  $\rightarrow$  Lenguaje 
    \item Gramáticas  $\rightarrow$  Automata  
    \item Gramáticas  $\rightarrow$  ER
\end{itemize}
\subsubsection{AutPila}
\begin{itemize}
    \item leng $\rightarrow$ pila $\checkmark$    
    \item Gr $\rightarrow$ pila 
    \item pila $\rightarrow$ Gr 
\end{itemize}
\subsubsection{Ejercicios tipo}
\begin{itemize}
    \item Bombeo
    \item Propiedades ER 
    \item Minimizar autómatas
    \item Pasar a Chomsky 
    \item Algoritmo Cocke-Younger-Kasami    
\end{itemize}
\newpage
\section{Conversiones}
\subsection{Lenguaje a ER}
\subsubsection{Enunciado}
A = \{1, 0\}
\newline
Construir una ER para palabras en las que el número de ceros es par
\subsubsection{Resolución}
Pensamos en palabras que puedan estar contenidas en este lenguaje: 
\newline 
111 0110 1010 1001 
\newline 
Es decir, puede empezar, o no, por un 1 y los 0 pueden estar separados por una cantidad 0 o ilimitada de 1. Luego, al igual que empieza, la palabra puede terminar, o no, por 1. Con esta idea, llegamos al siguiente autómata: 
\newline
\[1^*(01^*0)^*1^*\]
\newline
Pero tiene un problema. No nos permitiría palabras del estilo 00100. Para eso debería de alternarse entre los 0 la clausura de un 1 y ya no necesitamos la clausura final. Después de esto ya tenemos la solución: 
\subsubsection{Resultado}
\[1^*(01^*01^*)^*\]
\newpage
\subsubsection{Enunciado}
A = \{1, 0\}
\newline
Da una ER para palabras que contengan 1001 como secuencia
\subsubsection{Resolución}
Este es fácil. Se puede poner tantos 0 o 1 al principio y al final de esa secuencia. Eso nos deja con la siguiente expresión:
\subsubsection{Resultado}
\[(0 + 1)^*1001(0 + 1)^*\]
\newpage
\section{Pila}
\subsection{Lenguaje a AutPila}
\subsubsection{Enunciado}
Describir el autómata que acepte el siguiente lenguaje: 
\[L = \{0^i1^i : i > 0\}\]
Tanto por pila vacía como por estados finales. 
\subsubsection{Resolución}
Un autómata con pila está definido por 7 elementos:
\begin{itemize}
    \item $\delta$: Las transciciones del autómata 
    \item A: El alfabeto del autómata 
    \item B: El alfabeto de la pila 
    \item q: El conjunto de estados 
    \item F: El conjunto de estados finales 
    \item R: La letra incial de la pila 
    \item $q_0$: El estado inicial del autómata 
\end{itemize}
(Voy a hacerlo sin el grafo porque no me aptece ahora aprender a dibujar grafos en LaTeX)

\subsubsection{Resultado}
\textbf{Por pila vacía}
\begin{itemize}
    \item A = \{0,1\}
    \item B = \{R,X\}
    \item q = \{$q_0$,$q_1$\}
    \item F = \{$\epsilon$\}
\end{itemize}
\[\delta(q_0, R, 0) = \{(q_0, XR)\}\]
\[\delta(q_0, X, 0) = \{(q_0, XX)\}\]
\[\delta(q_0, R, \epsilon) = \{(q_0, \epsilon)\}\]
\[\delta(q_0, X, 1) = \{(q_1, \epsilon)\}\]
\[\delta(q_1, X, 1) = \{(q_1, \epsilon)\}\]
\[\delta(q_1, R, \epsilon) = \{(q_1, \epsilon)\}\]
\newline
\textbf{Por estado final}
Análogo al anterior pero dos de las transiciones cambian
\[\delta(q_0, R, \epsilon) = \{(q_3, \epsilon)\}\]
\[\delta(q_1, R, \epsilon) = \{(q_f, \epsilon)\}\]
Donde $q_f$ es el estado final
\newpage
\section{Ejercicios tipo}
\newpage
Buena suerte con el examen!!!
\end{document}

